\documentclass{hw_report}
\usepackage{booktabs}
\usepackage{rotating}
\usepackage{wrapfig}
\newcommand{\keff}{k_{\text{eff}}}

\onehalfspacing

\title{\textbf{Computer Project 1}\\ \vspace{1em}
NPRE 555: Reactor Theory I}
\author{Liam Pohlmann\\Nuclear, Plasma, \& Radiological Engineering, University of Illinois Urbana-Champaign\\\url{liamp2@illinois.edu}}

\date{\textbf{Due: 6 October 2025}}

\begin{document}
    \maketitle
%\tableofcontents
%\clearpage


    \section{Premise}
    The purpose of this paper is to outline the methods used to determine a $k$-eigenvalue solution along with a flux profile for a simple slab of two fissile material in 1-D cartesian with monoenergetic neutrons.
    A simple schematic of this can be seen in \cref{fig:problem-schematic}.
    \begin{wrapfigure}{r}{0.6\textwidth} % 'r' for right side, 'l' for left side
        \centering
        \vspace{-10pt} % optional: adjust vertical spacing
        \includegraphics[width=0.58\textwidth]{figs/problem-schematic.jpg}
        \caption{Problem schematic}
        \label{fig:problem-schematic}
        \vspace{-10pt} % optional: tighten space below the figure
    \end{wrapfigure}
    Though this problem is simple enough that a straightforward \texttt{Python} script could be created and generate good results, I proposed an additional constraint: the code must be able to calculate the $\keff$ and flux profile for arbitrary 1D cells (assuming no overlap).
    This additional goal dramatically changed the coding practices from script- to modular-oriented, where the focus has been on flexibility, clarity, and reusability.


    \section{Theory}

    The eigenvalue for each generation of neutrons is calculated with:
    \begin{equation}
        \label{eq:k-sample}
        k_\text{gen}=\frac{\text{Number of Neutrons Produced}}{\text{Number of Source Neutrons}}
    \end{equation}
    Then, for each additional generation, the next system $\keff$ value is estimated with:
    \begin{equation}
        \label{eq:keff}
        \keff \approx \sum_{i=1}^{N} k_{\text{gen},i}
    \end{equation}
    where $N$ is the number of $k_\text{gen}$'s that have been calculated.
    To get an estimate of the error of $\keff$, the variance is estimated.
    However, variance of $\keff$ must be calculated differently because it is the variance of \textit{mean}.
    To handle this special case, it can be shown~\cite{ 244MeanVariance} that the variance scales with an additional factor of $1/N$:
    \begin{equation}
        \label{eq:variance-of-the-mean-of-k}
        \sigma^2_k=\frac{1}{N(N-1)}\sum_{i=1}^{N} \left( k_i-\keff \right)^2
    \end{equation}
    This ensures that for each additional $k_\text{gen}$, the variance will shrink.

    For cell $i$, the flux has been estimated in two ways~\cite{lewisComputationalMethodsNeutron1993}.
    First, the collision estimate:
    \begin{equation}
        \label{eq:collision-estimate}
        \phi_i \approx \frac{C_i}{NV_i\Sigma^i_t}
    \end{equation}
    where $C_i$ is the number of collisions that occurred in cell $i$, $N$ is the total number of source particles, $V_i$ is the cell volume\footnote{For 1D, the cell volume is just the cell length.}, and $\Sigma^i_t$ is the total macroscopic cross section in cell $i$.
    Next, the path-length estimate:
    \begin{equation}
        \label{eq:path-length}
        \phi_i \approx \frac{l_i}{NV_i}
    \end{equation}
    where $N$ and $V_i$ are as before, and $l_i$ is the total length traversed through the cell be all neutrons in the simulation.


    \section{Code Design}
    A code was written in \texttt{C++} to facilitate the computations of arbitrary 1D regions either connected or separated by a void.
    It was designed to be as reusable and flexible as possible.
    Three major classes were created: \texttt{MCSlab}, \texttt{Region}, and \texttt{Neutron}.
    The first of these holds the major computational methods for $k$-eigenvalue and flux calculations, the second holds material and geometry information for each users-specified region, and the last holds information and methods pertaining to neutrons and their position within the simulation.
    A non-exhaustive list of code elements from each of these has been included along with a short description in \cref{tab: code classes}.
    \begin{sidewaystable}
        \centering
        \begin{tabular}{llll}
            \toprule
            \textbf{Class} & \textbf{Item}                   & \textbf{Type} & \textbf{Description}                                                                 \\
            \midrule
            MCSlab         & \texttt{k\_eigenvalue}          & Method        & Holds random walk algorithm, tally calculations                                      \\

            & \texttt{readInput }             & Method        & Read user-supplied input file and initialize attributes                              \\

            & \texttt{absorption}             & Method        & If absorption occurs, calculate number of neutrons born and kill particle            \\

            & \texttt{scatter}                & Method        & If scatter occurs, change the particle’s direction                                   \\

            & \texttt{calculateK}             & Method        & Calculate $k_{\text{gen}}$, $\keff$, and the standard deviation                      \\

            & \texttt{updatePathLengths}      & Method        & For every movement, add the flight distance of the neutron to bins                   \\

            & \texttt{updateCollisions}       & Method        & For every collision, add 1 to bin of collision location                              \\

            & \texttt{exportFlux}             & Method        & Export flux values to \texttt{.csv} file                                             \\
            Region         & \texttt{\_Sigma\_s}             & Attribute     & Scattering macroscopic cross section                                                 \\

            & \texttt{\_Sigma\_a}             & Attribute     & Absorption macroscopic cross section                                                 \\

            & \texttt{\_nu\_Sigma\_f}         & Attribute     & Fission macroscopic cross section multiplied by the number neutrons born per fission \\

            & \texttt{\_Sigma\_t}             & Attribute     & Total macroscopic cross section                                                      \\

            & \texttt{\_n\_per\_fission}      & Attribute     & Number of neutrons produced per absorption                                           \\

            & \texttt{populateCellLocs}       & Method        & Populate cell centers and bounds within a single region                              \\
            Neutron        & \texttt{movePositionAndRegion}  & Method        & Move neutron position and change its region accordingly                              \\

            & \texttt{setRandomStartPosition} & Method        & Set a random start position within fissionable material                              \\

            & \texttt{distanceToCollisions}   & Method        & Calculate the distance to next collision within region                               \\

            & \texttt{distanceToEdge}         & Method        & Calculate the distance to the nearest edge in the direction of $\mu$                 \\

            & \texttt{\_is\_alive}            & Attribute     & Boolean value holding whether particle is still alive                                \\

            & \texttt{kill}                   & Method        & Switches \texttt{\_is\_alive} to \texttt{False}, terminating the walk                \\
            \bottomrule
        \end{tabular}
        \caption{Major classes within code and their description}
        \label{tab: code classes}
    \end{sidewaystable}
    As recommended by the instructor, each class also has an associated test created using the \texttt{gtest} framework.
    These tests are simple unit tests that were an integral part of the debugging and implementation process.

    In addition, a flowchart has been provided in \cref{fig:flowchart} that outlines the steps taken in a single generation of neutrons.
    It was heavily inspired through lecture notes given in class~\cite{huffLectureNotes2025}.
    Note that after each generation, the Shannon-Entropy, running $\keff$ value, and standard deviation are reported to the user as commandline output.

    \begin{figure}[!htb]
        \centering
        \includegraphics[width=0.8\textwidth]{figs/MC-flowchart}
        \caption{Monte Carlo code flowchart for a single generation}
        \label{fig:flowchart}
    \end{figure}


    \section{Results}
    For the requested problem shown in \cref{fig:problem-schematic}, \cref{fig:submission_combined} has been included.
    It reports the flux profiles using both \cref{eq:collision-estimate} and \cref{eq:path-length} along with the estimate of $\keff$ and its standard-deviation.
    \begin{figure}[!htb]
        \centering
        \begin{subfigure}[b]{0.48\textwidth}
            \centering
            \includegraphics[width=\textwidth]{../MCSlab/examples/plot_submission_simulation_pl}
            \caption{Path length estimate}
            \label{submission_path_length}
        \end{subfigure}
        \hfill
        \begin{subfigure}[b]{0.48\textwidth}
            \centering
            \includegraphics[width=\textwidth]{../MCSlab/examples/plot_submission_simulation_col}
            \caption{Collision estimate}
            \label{submission_collision}
        \end{subfigure}
        \caption{Comparison of simulation estimates. $k_{\text{eff}}=1.183377 \pm 0.000328$}
        \label{fig:submission_combined}
    \end{figure}
    As seen, both methods produce extremely similar results, differing only in the magnitude (which can be solved with a normalization).
    \texttt{README} files have been included in the GitHub repository that will allow the user to replicate these results.


    \section{Future Work}
    Though tests were provided and successful, additional tests would be worthwhile to ensure complete functionality.
    Additionally, a regression test of sorts could be implemented for a simple analytically-known problem to verify the code works as intended on each machine.
    Lastly, due to time constraints, there is still some work to be done to ensure proper documentation in the header files.

    \bibliographystyle{plain}
    \bibliography{CP1}

    \clearpage

    \appendix


    \section{Additional Example Solutions}
    To prove the success of the code, three additional simulations were run as example cases.
    They demonstrate the ability of the code to handle single and multi-region calculations and transport neutrons across voids between regions where necessary.
    \begin{enumerate}
        \item Single region: \cref{fig:single_combined}
        \item Connected double region: \cref{fig:double-connected-combined}
        \item Separated double region: \cref{fig:split-path-length}
    \end{enumerate}

    \begin{figure}[!htb]
        \centering
        \begin{subfigure}[b]{0.48\textwidth}
            \centering
            \includegraphics[width=\textwidth]{../MCSlab/examples/plot_1_region_pl}
            \caption{Single region path length estimate}
            \label{single_path_length}
        \end{subfigure}
        \hfill
        \begin{subfigure}[b]{0.48\textwidth}
            \centering
            \includegraphics[width=\textwidth]{../MCSlab/examples/plot_1_region_col}
            \caption{Single region collision estimate}
            \label{single_collision}
        \end{subfigure}
        \caption{Comparison of simulation estimates.}
        \label{fig:single_combined}
    \end{figure}

    \begin{figure}[!htb]
        \centering
        \begin{subfigure}[b]{0.48\textwidth}
            \centering
            \includegraphics[width=\textwidth]{../MCSlab/examples/plot_2_region_connected_pl}
            \caption{Double region connected path length estimate}
            \label{double_connected_path_length}
        \end{subfigure}
        \hfill
        \begin{subfigure}[b]{0.48\textwidth}
            \centering
            \includegraphics[width=\textwidth]{../MCSlab/examples/plot_2_region_connected_col}
            \caption{Double region connected collision estimate}
            \label{double_connected_collision}
        \end{subfigure}
        \caption{Comparison of simulation estimates.}
        \label{fig:double-connected-combined}
    \end{figure}


    \begin{figure}[!htb]
        \centering
        \includegraphics[width=\textwidth]{../MCSlab/examples/plot_2_region_split_pl}
        \caption{Double region split path length estimate}
        \label{fig:split-path-length}
    \end{figure}

\end{document}
