\documentclass{hw_report}
\usepackage{booktabs}
\usepackage{rotating}
\usepackage{wrapfig}
\usepackage{physics}
\newcommand{\keff}{k_{\text{eff}}}

\onehalfspacing

\title{\textbf{Computer Project 3}\\ \vspace{1em}
NPRE 555: Reactor Theory I}
\author{Liam Pohlmann\\Nuclear, Plasma, \& Radiological Engineering, University of Illinois Urbana-Champaign\\\url{liamp2@illinois.edu}}

\date{\textbf{Due: 6 October 2025}}

\begin{document}
    \maketitle
%\tableofcontents
%\clearpage


    \section{Premise}
    Though Monte Carlo methods are considered the gold-standard for neutron transport, analog Monte Carlo struggles greatly to give accurate (low-variance) solutions in regions where neutron count is low.
    Such applications include radiation shields and regions far from the core of the reactor.
    Local variance reduction techniques are easily employed for source-detector problems using adjoint-informed importance maps.
    However, these methods restrict their increased accuracy to only a small region of interest.
    For global problems, reducing the variance becomes a bigger issue.

    The Forward-Weighted Consistent Adjoint Driven Importance Sampling method (FW-CADIS) is a technique that first emerged in the late 1990's by John Wagner~\cite{wagnerAutomatedVarianceReduction1998} for fixed-source problems.
    As will be discussed in the next section, the method looks to create an importance map for Monte Carlo methods by first approximating the global neutron flux with a computationally-cheap deterministic solve, then using this forward solution as the source for an adjoint problem, and finally using this solution to create target weights for weight windows.
    This method was attempted to be applied to 1D geometry on both a simple and challenge problem.


    \section{Theory}
    In analog Monte Carlo as introduced in Lewis and Miller's textbook~\cite{lewisComputationalMethodsNeutron1993}, each neutron is treated as a single particle.
    When looking to variance reduction (VR) techniques, the notion of particle ``weight'' is introduced, wherein a single computational particle may in reality represent more (or less) than a single physical particle.
    This idea allows us to hold on to more physical particles we deem important.

    Weight windows are a population control VR technique that allows us to specify the weight of computer particles that enter a region.
    If a particle is below the window, we play a game of Russian roulette, where we throw a random number between 0 and 1 and compare this to the probability of survival, defined as:
    \begin{equation}
        \label{p survive}
        p_{survival}=\frac{w_{neutron}}{w_{l}}
    \end{equation}
    where $w_l$ is the lower bound of the weight window.
    If the particle survives, its weight is bumped up to the center weight of the window.
    This permits more particles of higher weight to continue deeper into the domain.
    If a particle's weight is \textit{greater} than that of the upper window bounds, then the particle splits, creating more computational particles with a fraction of the weight of the original.~\cite{TechReport_2024_LANL_LA-UR-24-24602Rev.1_KuleszaAdamsEtAl}
    This allows for more unique histories to be run rather than a single history with a high weight, giving more information to global tallies such as the flux.

    For simplicity, constant-value nuclear data is chosen.
    This greatly simplifies the transport equation to:
    \begin{equation}
        \label{forward bte}
        \mu \pdv[]{\psi}{x}+\Sigma_t \psi = \frac{\Sigma_s}{2}\int_{-1}^{1} \psi(x,\mu)\dd\mu +\frac{1}{k}\frac{\nu\Sigma_f}{2}\int_{-1}^{1} \psi(x,\mu)\dd\mu+Q(x,\mu)
    \end{equation}
    A keen observer will notice that $\int_{-1}^{1} \psi(x,\mu)\dd\mu$ is just the scalar flux.

    In the FW-CADIS method, we first solve \cref{forward bte} using a cheap deterministic solution\footnote{In practice, any method is fine as long as it is cheap and quick.} to get a good-enough estimate of the forward flux.
    Then, we solve the fixed-source adjoint problem, defined by:
    \begin{equation}
        \label{adjoint}
-\mu \pdv[]{\psi^{\dagger}}{x}+\Sigma_t \psi^{\dagger} = \frac{\Sigma_s}{2}\int_{-1}^{1} \psi^{\dagger}(x,\mu)\dd\mu +q^{\dagger}(x)
    \end{equation}
    where:
    \begin{equation}
        \label{fwcadis source}
        q^{\dagger}(x)=\frac{1}{\phi(x)}
    \end{equation}
    Using the definition from Wagner~\cite{wagnerAutomatedVarianceReduction1998}, we can set our weight window parameters:
    \begin{subequations}
        \label{ww vals}
        \begin{equation}
            \label{lower ww}
            w_l=\left[ \phi^\dagger \left( \frac{c+1}{2} \right) \right]^{-1}
        \end{equation}
        \begin{equation}
            \label{target}
            w_t=\frac{w_u+w_l}{2}
        \end{equation}
        \begin{equation}
            \label{width}
            c=\frac{w_u}{w_l}
        \end{equation}
    \end{subequations}
    where $c$ is a user-set parameter known as the ``window width,'' $w_t$ is the target (or center) weight, and $w_u$ is the upper weight.
    Then the Monte Carlo game is played using weight windowing for population control


    \section{Results}
    To start, the simple problem introduced in Computer Project 1 was implemented using the FW-CADIS algorithm.
    The specifications for this simulation are as follows:
    \begin{itemize}
        \item \textbf{Regions:}
        \begin{itemize}
            \item Region 1: $x \in [0, 50]$, $n_\text{cells}=50$, $\Sigma_a=0.12$, $\Sigma_s=0.05$, $\nu\Sigma_f=0.15$, source=0
            \item Region 2: $x \in [50, 100]$, $n_\text{cells}=50$, $\Sigma_a=0.10$, $\Sigma_s=0.05$, $\nu\Sigma_f=0.12$, source=0
        \end{itemize}
        \item \textbf{Simulation Settings:}
        \begin{itemize}
            \item Number of particles: 20,000
            \item Number of generations: 25
            \item Number of inactive generations: 10
            \item Variance reduction method: FW-CADIS
            \item Quadrature order: 8
            \item Weight window width: 4
        \end{itemize}
    \end{itemize}
    This produced a $k_{eff}$ of $1.202194 \pm 0.001480$.
    The full output is given in \cref{sec:cp1-out}.

    \begin{figure}[!htb]
        \centering
        \begin{subfigure}[b]{0.48\textwidth}
            \centering
            \includegraphics[width=\textwidth]{figs/cp1/submission_simulation_collsion_from_bins}
            \caption{CP1 problem flux using collision estimate.}
        \end{subfigure}
        \hfill
        \begin{subfigure}[b]{0.48\textwidth}
            \centering
            \includegraphics[width=\textwidth]{figs/cp1/submission_simulation_pathlength_from_bins}
            \caption{CP1 problem flux using pathlength estimate.}
        \end{subfigure}
        \hfill
        \begin{subfigure}[b]{0.48\textwidth}
            \centering
            \includegraphics[width=\textwidth]{figs/cp1/submission_simulation_SN_plot}
            \caption{CP1 problem flux estimated with discrete ordinates (8-point quadrature). Also plotted is the adjoint solution with the inverse of the forward flux and the applied importance map.}
        \end{subfigure}
        \hfill
        \begin{subfigure}[b]{0.48\textwidth}
            \centering
            \includegraphics[width=\textwidth]{figs/cp1/submission_simulation_relative_unc}
            \caption{Relative uncertainty of Monte Carlo calculations.}
        \end{subfigure}
        \caption{Results of FW-CADIS method on CP1 computer problem.}
        \label{fig:cp1-results}
    \end{figure}

    In addition to this problem, a challenge problem was constructed to attempt to test the variance reduction from the method.
    A schematic of this is given in \cref{fig:challenge-prob-schematic}.

    \begin{figure}[!htb]
        \centering
        \includegraphics[width=0.8\textwidth]{figs/test_prob_figure}
        \caption{Challenge problem schematic.}
        \label{fig:challenge-prob-schematic}
    \end{figure}

    The run parameters are specified as:
    \begin{itemize}
        \item \textbf{Regions:}
        \begin{itemize}
            \item Region 1: $x \in [-10.0, -7.5]$, $n_\text{cells}=20$, $\Sigma_a=0$, $\Sigma_s=1$, $\nu\Sigma_f=0$, source=0
            \item Region 2: $x \in [-7.5, -5.0]$, $n_\text{cells}=20$, $\Sigma_a=0.5$, $\Sigma_s=0$, $\nu\Sigma_f=0.1$, source=0
            \item Region 3: $x \in [-5.0, 5.0]$, $n_\text{cells}=20$, $\Sigma_a=0.05$, $\Sigma_s=0$, $\nu\Sigma_f=0.12$, source=0
            \item Region 4: $x \in [5.0, 7.5]$, $n_\text{cells}=20$, $\Sigma_a=0.5$, $\Sigma_s=0$, $\nu\Sigma_f=0.1$, source=0
            \item Region 5: $x \in [7.5, 10.0]$, $n_\text{cells}=20$, $\Sigma_a=0$, $\Sigma_s=1$, $\nu\Sigma_f=0$, source=0
        \end{itemize}
        \item \textbf{Simulation Settings:}
        \begin{itemize}
            \item Number of particles: 15,000
            \item Number of generations: 20
            \item Number of inactive generations: 5
            \item Variance reduction method: FW-CADIS
            \item Quadrature order: 12
            \item Weight window width: 2
        \end{itemize}
    \end{itemize}

    Results of the runs are seen in \cref{fig:challenge-results} and \cref{sec:challenge-out}.
    \begin{figure}[!htb]
        \centering
        \begin{subfigure}[b]{0.48\textwidth}
            \centering
            \includegraphics[width=\textwidth]{figs/challenge/challenge_prob_SN_plot}
            \caption{Challenge problem flux using collision estimate.}
        \end{subfigure}
        \hfill
        \begin{subfigure}[b]{0.48\textwidth}
            \centering
            \includegraphics[width=\textwidth]{figs/challenge/challenge_prob_collsion_from_bins}
            \caption{Challenge problem flux using pathlength estimate.}
        \end{subfigure}
        \hfill
        \begin{subfigure}[b]{0.48\textwidth}
            \centering
            \includegraphics[width=\textwidth]{figs/challenge/challenge_prob_pathlength_from_bins}
            \caption{Challenge problem flux estimated with discrete ordinates (12-point quadrature). Also plotted is the adjoint solution with the inverse of the forward flux and the applied importance map.}
        \end{subfigure}
        \hfill
        \begin{subfigure}[b]{0.48\textwidth}
            \centering
            \includegraphics[width=\textwidth]{figs/challenge/challenge_prob_relative_error}
            \caption{Relative uncertainty of Monte Carlo calculations.}
        \end{subfigure}
        \caption{Results of FW-CADIS method on formulated challenge problem.}
        \label{fig:challenge-results}
    \end{figure}


    \section{Discussion}
    What was discovered far too late into this project is that FW-CADIS was not originally intended to be used for fission problems.
    Indeed, as can be seen in the discrete-ordinates results above, the importance maps are effectively flipped from what they should be.
    This is what caused the flux profiles of the CP1 problem (as seen in \cref{fig:cp1-results}) to be mirrored.
    I attempted to remedy this by using the inverse of the raw adjoint solve (no forward solve, just a pure fission-driven adjoint), but the results were subpar.
    Results for this are seen in \cref{adjoint only}.
    \begin{figure}[!htb]
        \centering
        \includegraphics[width=0.8\textwidth]{figs/adjoint/submission_simulation_pathlength_from_bins}
        \caption{CP1 with adjoint-only weight windows.}
        \label{adjoint only}
    \end{figure}
    Note that in this attempt, the Monte Carlo simulation converged to the correct value of $1.219136 \pm 0.000867$, but the flux shape is problematic.

    Additionally, code was written to compute the integrated variance\footnote{This is done in post-processing.}, and the intention of this was to do a figure-of-merit study as done by Wagner.
    However, because the results are incorrect, this study would be meaningless.
    Had I been able to redo this problem, I would have chosen to instead attempt the MAGIC method or multiple VR techniques such as implicit capture\footnote{This was attempted to implement with FW-CADIS but unsuccessful.}.

    \bibliographystyle{ieeetr}
    \bibliography{CP3}

    \clearpage
    \appendix
    \section{CP1 Problem Output}\label{sec:cp1-out}
    \inputminted[linenos, bgcolor=LightGray, fontsize=\footnotesize]{text}{output/sub_out.txt}
    \section{Challenge Problem Output}\label{sec:challenge-out}
    \inputminted[linenos, bgcolor=LightGray, fontsize=\footnotesize]{text}{output/challenge_out.txt}
\end{document}
